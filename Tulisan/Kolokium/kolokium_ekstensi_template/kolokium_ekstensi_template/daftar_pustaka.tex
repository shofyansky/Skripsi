\begin{thebibliography}{100} % 100 is a random guess of the total number of references
% ----- Tuliskan tiap pustaka yang diacu seperti contoh di bawah ini
% ----- Tiap pustaka dimulai dengan \bibiitem{id-pustaka}
\bibitem{Adinugroho2005} Adinugroho WC, Suryadiputra INN, Saharjo BH, Siboro L. 2005. Panduan Pengendalian Kebakaran Hutan dan Lahan Gambut. Proyek Climate Changes, Forests and Peatlands in Indonesia. Bogor(ID): Wetlands International-Indonesia Programme and Wildlife Habibat Canada.

\bibitem{Baehaki2014}Baehaki D. 2014. Deteksi pencilan data titik panas di provinsi Riau menggunakan algoritme  clustering K-Means [skripsi]. Bogor(ID): Insitut Pertanian Bogor.

\bibitem{Beunig2000}Beunig Markus M, Kriegel Hans-Peter, Ng Raymond T, Sander J. 2000. LOF: Identifying Density-Based Local Outliers. ACM SIGMOD international conference on Management  of data; 2, June 2000; New York, USA. New York (USA): ACM SIGMOD Volume 29 Issue Pages 93-104 

\bibitem{Cahyadarena2014}Cahyadarena M B.2014. Deteksi Pencilan Pada Data Titik Panas Menggunakan Clustering Berbasis Medoids [skripsi]. Bogor(ID): Insitut Pertanian Bogor.

\bibitem{Guswanto2009}Guswanto, Heriyanto E. 2009. Operational Weather System for National Fire Danger Rating. Jurnal Meteorologi dan Geofisika. 10(2): 77-87

\bibitem{Han2012}Han J, Kamber M, Pei J. 2012.  Data mining: concepts and  techniques.  Massachusetts (US) : Morgan Kaufmann.

\bibitem{Hasan1999}Hasan M I. 1999. Pokok-Pokok Materi Statistik 1: Statistik Deskriptif. Jakarta (ID): Bumi aksara.

\bibitem{Sizer2013}Sizer N, Anderson J, Stolle F, Minnemeyer S, Higgins M, Leach A, Alisjahbana A, Utami A. 2014. Kebakaran Hutan di Indonesia Mencapai Tingkat Tertinggi Sejak Kondisi Darurat Kabut Asap Juni 2013 [Internet]. [diunduh 2015 17 Mei]. Tersedia pada http://www.wri.org/blog/2014/03/kebakaran-hutan-di-indonesia-mencapai-tingkat-  tertinggi-sejak-kondisi-darurat-kabut. 

\bibitem{Suwarsono2013}Suwarsono, Rokhmatuloh, Waryono T. 2013. Pengembangan Model Identifikasi Daerah Bekas Kebakaran Hutan dan Lahan (Burned Area) Menggunakan Citra MODIS di Kalimantan [Model Development of Burned Area Identification Using MODIS Imagery in Kalimantan]. Jurnal Penginderaan Jauh. 10(2): 93-112.

\bibitem{Tacconi2003}Tacconi L. 2003. Kebakaran Hutan di Indonesia: Penyebab, Biaya dan Implikasi  Kebijakan[paper]. Bogor(ID): Center For International Forestry Research


% Perintah ini digunakan untuk membuat column break (jika diperlukan)
\vfill\eject


% ----- Akhir dari pustaka
\end{thebibliography}
