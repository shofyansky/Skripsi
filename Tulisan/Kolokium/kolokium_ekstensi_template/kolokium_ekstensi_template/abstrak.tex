%----------------------------------------------------------------------------------------
%	ABSTRACT
%----------------------------------------------------------------------------------------
\Abstract{\scriptsize 
% ---- Tuliskan abstrak di bagian ini seperti contoh.
Indonesia mengalami kebakaran hutan yang signifikan. Pada tahun 2013 World Resources Institute (WRI)  meneliti tren historis titik panas di Pulau Sumatera menggunakan data titik panas aktif National Aeronautics and Space Administration (NASA). Pada 13-30 Juni 2013 terjadi 2643 total jumlah peringatan titik panas. Tahun berikutnya Pada  20 Februari hingga 11 Maret tahun 2014  titik panas meningkat menjadi 3101 peringatan titik panas. Salah satu upaya untuk menangani kebakaran hutan ialah dengan menganalisis data titik panas yaitu dengan menganalisis pencilan titik panas sehingga dapat diidentifikasi wilayah yang beresiko terjadinya kebakaran hutan. Beberapa penelitian terkait deteksi pencilan yang sudah dilakukan diantaranya menggunakan algoritme clustering k-means dan juga menggunakan algoritme clustering berbasis medoids. Kedua penelitian tersebut mendeteksi pencilan berdasarkan frekuensi terjadinya titik panas dan belum mendeteksi pencilan berdasarkan kepadatan penyebaran titik panas. Algoritme yang dapat mendeteksi pencilan berdasarkan kepadatan penyebaran titik panas ialah algoritme local outlier factor.  Dengan algoritme local outlier factor informasi mengenai wilayah yang berpotensi terjadi kebakaran hutan berdasarkan kepadatan penyebaran titik panas dapat dideteksi sehingga menjadi informasi tambahan untuk pengambilan keputusan oleh pihak terkait.
% ---- Akhir bagian abstrak
\normalsize}
